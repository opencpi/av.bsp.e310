\iffalse
This file is protected by Copyright. Please refer to the COPYRIGHT file
distributed with this source distribution.

This file is part of OpenCPI <http://www.opencpi.org>

OpenCPI is free software: you can redistribute it and/or modify it under the
terms of the GNU Lesser General Public License as published by the Free Software
Foundation, either version 3 of the License, or (at your option) any later
version.

OpenCPI is distributed in the hope that it will be useful, but WITHOUT ANY
WARRANTY; without even the implied warranty of MERCHANTABILITY or FITNESS FOR A
PARTICULAR PURPOSE. See the GNU Lesser General Public License for more details.

You should have received a copy of the GNU Lesser General Public License along
with this program. If not, see <http://www.gnu.org/licenses/>.
\fi

\iffalse

This snippet defines macros to be used when describing ocpidev commands
to give the user the equivalent in the IDE. See AV-4628.

To see all output:
\code{\$ ocpidev build something something}
\OcpidevBuild
\code{\$ ocpidev clean something something}
\OcpidevClean
\code{\$ ocpidev run test something something}
\OcpidevRunTest
\code{\$ ocpidev create (no options)}
\OcpidevCreate{}
\code{\$ ocpidev create project}
\OcpidevCreate{Project}
\code{\$ ocpidev clean project Project}
\OcpidevCleanProject{Project}
\code{\$ ocpidev register project my\_proj}
\OcpidevRegisterProject{my_proj}
\code{\$ ocpidev unregister project my\_proj}
\OcpidevUnRegisterProject{my_proj}

\fi
% https://tex.stackexchange.com/a/5227
\usetikzlibrary{shadows}
\newcommand*\OcpidevKeystroke[1]{%
  \tikz[baseline=(key.base)]
    \node[%
      draw,
      fill=white,
      drop shadow={shadow xshift=0.25ex,shadow yshift=-0.25ex,fill=black,opacity=0.75},
      rectangle,
      rounded corners=2pt,
      inner sep=1pt,
      line width=0.5pt,
      font=\scriptsize\sffamily
    ](key) {#1\strut}
  ;
}

\providecommand{\OcpidevCtrlClick}{(use \OcpidevKeystroke{~Ctrl~} for multiple selection)}

\providecommand{\OcpidevTemplate}[1]{
\begin{center}
\framebox{\parbox{0.8\linewidth}{\textit{To perform this operation within the IDE:}
#1}}
\end{center}
}

% OcpidevBuild = "ocpidev build"
\providecommand{\OcpidevBuild}{\OcpidevTemplate{
\begin{enumerate}
\setlength\itemsep{0em} %tighten
\item Open the ANGRYVIPER Perspective
\item Select the asset from OpenCPI Project View
\item Import to AV Operations Panel using ``$>$'' button
\item Select the RCC and/or HDL platforms for the build \OcpidevCtrlClick
\item Click ``Build''
\end{enumerate}
% Open the ANGRYVIPER Perspective, select the RCC and/or HDL platforms for the build \OcpidevCtrlClick , then select the asset in OpenCPI Projects view, right click, select build.
}}

% OcpidevClean = "ocpidev clean"
\providecommand{\OcpidevClean}{\OcpidevTemplate{
In the OpenCPI Projects view, select the project, right-click, select clean from the menu.
}}

% OcpidevRunTest = "ocpidev run test"
\providecommand{\OcpidevRunTest}{\OcpidevTemplate{
\begin{enumerate}
\setlength\itemsep{0em} %tighten
\item Click the ``Tests'' radio button and select RCC and/or HDL platforms for the run \OcpidevCtrlClick .
\item Click the ``+remotes'' button, enter the remote string, click OK.
\item Select the remote in the remotes list.
\item In the OpenCPI Projects view, select the desired unit tests, click the ``$>$'' button in the operations panel, then click the desired operation to build and then run the listed tests on the selected remote.
\end{enumerate}
}}

% OcpidevCreate = "ocpidev create <$1>"
\providecommand{\OcpidevCreate}[1]{\OcpidevTemplate{
\begin{itemize}
\setlength\itemsep{0em} %tighten
\item Place the cursor in the OpenCPI Projects panel, right click, select asset wizard.
\item Select the asset type\ifthenelse{\equal{#1}{}}{}{ (``#1'')} in the drop-down, fill in the required inputs, click finish.
\item When the process finishes, the new asset is displayed in both project views. (If the asset has an XML editor, then the editor opens.)
\end{itemize}
}}

% OcpidevCleanProject = "ocpidev clean project <$1>"
\providecommand{\OcpidevCleanProject}[1]{\OcpidevTemplate{\textit{(The project ``#1'' must be imported into the IDE and then refresh the OpenCPI Projects view so the project is shown.)}
\begin{itemize}
\setlength\itemsep{0em} %tighten
\item Right click on #1 $\Rightarrow$ ``Clean''
\end{itemize}
}}

% OcpidevRegisterProject = "ocpidev register project <$1>"
% OcpidevUnRegisterProject = "ocpidev unregister project <$1>"
\providecommand{\OcpidevRegisterProjectKernel}[2]{\OcpidevTemplate{\textit{(The project ``#1'' must be imported into the IDE and then refresh the OpenCPI Projects view so the project is shown.)}
\begin{itemize}
\setlength\itemsep{0em} %tighten
\item In the OpenCPI Projects view, select the project, right-click, select ``#2'' from the menu. (Depending on state of the project, this option may not be available.)
\end{itemize}
}}
\providecommand{\OcpidevRegisterProject}[2]{\OcpidevRegisterProjectKernel{\path{#1}}{register}}
\providecommand{\OcpidevUnRegisterProject}[2]{\OcpidevRegisterProjectKernel{\path{#1}}{unregister}}
