\iffalse
This file is protected by Copyright. Please refer to the COPYRIGHT file
distributed with this source distribution.

This file is part of OpenCPI <http://www.opencpi.org>

OpenCPI is free software: you can redistribute it and/or modify it under the
terms of the GNU Lesser General Public License as published by the Free Software
Foundation, either version 3 of the License, or (at your option) any later
version.

OpenCPI is distributed in the hope that it will be useful, but WITHOUT ANY
WARRANTY; without even the implied warranty of MERCHANTABILITY or FITNESS FOR A
PARTICULAR PURPOSE. See the GNU Lesser General Public License for more details.

You should have received a copy of the GNU Lesser General Public License along
with this program. If not, see <http://www.gnu.org/licenses/>.
\fi

% This is for inserting into various "Getting Started" Guides
% First, turn off indenting to avoid all the flushleft
\newlength{\savedparindentsystime}%
\setlength{\savedparindentsystime}{\parindent}%
\setlength{\parindent}{0pt} % Don't indent all paragraphs
\providecommand{\forceindent}{\leavevmode{\parindent=1em\indent}}%
\subsection{Setup system time reference}
\label{sec:Setup system time reference}
\textbf{If Linux system time is not required to be accurate, this step may be skipped.} \\ \medskip

\textit{For either Network or Standalone mode}, the following settings that are passed by \texttt{mynetsetup.sh/mysetup.sh} to the \texttt{zynq\_net\_setup.sh/zynq\_setup.sh} scripts \textit{may} require modification:

\begin{itemize}
 \item Identify the system that is to be used as a time server, where the default is ``time.nist.gov'' and is set in \path{/mnt/card/opencpi/ntp.conf}. A valid time server must support ntp.
 \item Identify the current timezone description, where the default is ``EST5EDT,M3.2.0,M11.1.0''. Change this if required for the local timezone. See \texttt{man tzset} on the host PC for more information.
 \item If a time server is not required, or cannot connect to a time server, the user is required to manually set the time at start up.  Use the \code{date} command to manually set the Linux system time. See \texttt{man date} on the host PC for more information.
\end{itemize}
\setlength{\parindent}{\savedparindentsystime}%
