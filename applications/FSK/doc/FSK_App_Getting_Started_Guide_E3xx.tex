\iffalse
This file is protected by Copyright. Please refer to the COPYRIGHT file
distributed with this source distribution.

This file is part of OpenCPI <http://www.opencpi.org>

OpenCPI is free software: you can redistribute it and/or modify it under the
terms of the GNU Lesser General Public License as published by the Free Software
Foundation, either version 3 of the License, or (at your option) any later
version.

OpenCPI is distributed in the hope that it will be useful, but WITHOUT ANY
WARRANTY; without even the implied warranty of MERCHANTABILITY or FITNESS FOR A
PARTICULAR PURPOSE. See the GNU Lesser General Public License for more details.

You should have received a copy of the GNU Lesser General Public License along
with this program. If not, see <http://www.gnu.org/licenses/>.
\fi

%----------------------------------------------------------------------------------------
% Update the docTitle and docVersion per document
%----------------------------------------------------------------------------------------
\def\docTitle{OpenCPI\\ FSK App Getting Started Guide\\ (E310 Supplement)}
\def\docVersion{1.4}
%----------------------------------------------------------------------------------------
\input{../../../snippets/LaTeX_Header.tex}
\date{Version \docVersion} % Force date to be blank and override date with version
\title{\docTitle}
\lhead{FSK App Getting Started Guide}
%----------------------------------------------------------------------------------------
\usepackage[T1]{fontenc} % http://tex.stackexchange.com/a/181119
\usepackage{graphicx}
\graphicspath{ {figures/} }
\usepackage{textcomp}
\begin{document}
\maketitle
%\thispagestyle{fancy}
\newpage

	\begin{center}
	\textit{\textbf{Revision History}}
		\begin{table}[H]
		\label{table:revisions} % Add "[H]" to force placement of table
			\begin{tabularx}{\textwidth}{|c|X|l|}
			\hline
			\rowcolor{blue}
			\textbf{Revision} & \textbf{Description of Change} & \textbf{Date} \\
		    \hline
		    v1.1 & Initial Release & 3/2017 \\
		    \hline
		    v1.2 & Updated for OpenCPI Release 1.2 & 8/2017 \\
			\hline
			v1.3 & Updated for OpenCPI Release 1.3 & 1/2018 \\
			\hline
			v1.3.1-E3XX & Updated for E310 support & 3/2018 \\
			\hline
			v1.4 & Updated with simplications and references to assets' document & 9/2018 \\
			\hline
			\end{tabularx}
		\end{table}
	\end{center}

\newpage

\tableofcontents

\def\assetsdoc{\noindent For more information on this application, see \code{ocpi.assets}'s more in-depth version of the \textit{FSK\_app} document.}

\newpage

\section{References}

	This document assumes a basic understanding of the Linux command line (or ``shell'') environment.  The reference(s) in Table 1 can be used as an overview of OpenCPI and may prove useful.

\def\myreferences{
\hline
FSK App\footnote{Provides details of the ``FSK App'' reference application} & OpenCPI & \path{FSK_app.pdf}\\
}
\input{../../../snippets/References_Table}

\newpage
\begin{flushleft}
\section{Overview}
The purpose of this document is to provide a compact set of instructions to build, run, and verify the OpenCPI FSK App reference application.

\section{Prerequisites}
This document assumes that the OpenCPI framework has been installed. The application is supported on the Ettus E310 platform.

\section{Build the Core Project}
If the Core Project has not been created yet, follow the instructions in the OpenCPI Getting Started Guide. Once the Core project has been created, the following ocpidev command can be used to build the primitives and workers required by the FSK app. Navigate to the Core project directory and run the command:
\begin{verbatim}
ocpidev build --rcc --rcc-platform xilinx13_4 --hdl --hdl-platform e3xx --no-assemblies
\end{verbatim}

Note: The --no-assemblies argument excludes the creation of executable bitstreams.\\
This step takes approximately 20 minutes to complete.\\

\section{Build the Assets Project}
If the Assets Project has not been created yet, follow the instructions in the OpenCPI Getting Started Guide. Once the Assets project has been created, the following ocpidev command can be used to build the primitives and workers required by the FSK app. Navigate to the Assets project directory and run the command:
\begin{verbatim}
ocpidev build --rcc --rcc-platform xilinx13_4 --hdl --hdl-platform e3xx --no-assemblies
\end{verbatim}
Note: The --no-assemblies is necessary here because we cannot build assemblies for the e3xx platform until that platform has itself been built.\\

This step takes approximately 50 minutes to complete.

\section{Build the E310 BSP Project}
For the Ettus E310, there is a BSP project that contains a copy of the FSK assemblies, so you can build it \textit{after} the assets project. Navigate to the E310 BSP project directory and run the following command:
\begin{verbatim}
ocpidev build --rcc --rcc-platform xilinx13_4 --hdl --hdl-platform e3xx
\end{verbatim}
This step takes approximately 45 minutes to complete.

\section{Build the Assets for E310}
Now that the E310 platform has been built, we can return to the assets project and build the fsk\_filerw assembly for the E310 platform:
\begin{verbatim}
ocpidev build hdl assembly fsk_filerw --hdl-platform e3xx
\end{verbatim}
This step takes approximately 10 minutes to complete.


\section{Build the FSK Application Executable}
Next, the executable for the FSK Application must be built. Navigate to the E310 BSP project's \texttt{applications/FSK} directory of the Assets project and run the command:
\begin{verbatim}
ocpidev build --rcc-platform xilinx13_4
\end{verbatim}
	If successful, a new directory named \code{target-xilinx13\_4} will contain the executable.

\section{Running the Application}
For more information, see the full \textit{FSK\_App\_Getting\_Started\_Guide} in the Assets project. You can also reference the E310 \textit{FSK\_app} document, or run ``\code{make show}'' (on the host) in the E310 BSP's \texttt{applications/FSK} directory for more E310 specific information (especially \code{OCPI\_LIBRARY\_PATH} settings, etc...).\\\medskip

In short, once the radio is set up, run the application \textit{on the embedded radio} by running the executable and passing a ``mode'' such as \code{filerw}, \code{tx}, \code{rx} or \code{txrx}. For example:\\\medskip
Connect port TRXA to RX2A using an SMA connector, enter the E310 BSP project's \texttt{applications/FSK} directory, and run the following:\\
\begin{verbatim}
./target-xilinx13_4/FSK txrx
\end{verbatim}
At the runtime, choose TX port: TRXA and RX port: RX2A. The default values should suffice for the remaining options.
\subsection{View the Results}
After the application completes, the output can be found in the \texttt{applications/FSK/odata} directory. To view the results on the Development Host, navigate to the E310 BSP project's \texttt{applications/FSK} directory and execute the command:
\begin{verbatim}
eog odata/out_app_fsk_txrx.bin
\end{verbatim}
\end{flushleft}
	\begin{figure}[ht]
	 	\centering
	 	\begin{minipage}{.325\textwidth}
			\centering\includegraphics[width=1.0\linewidth]{Os}
			\caption{FSK input file}
			\label{fig:os_pic}
		\end{minipage}
	 	\begin{minipage}{.45\textwidth}
			\centering\includegraphics[width=1.0\linewidth]{tx_spec_an}
			\caption{Output of FSK App RF transmit}
			\label{fig:tx_spec_an}
		\end{minipage}
	\end{figure}
\pagebreak
\end{document}
