\iffalse
This file is protected by Copyright. Please refer to the COPYRIGHT file
distributed with this source distribution.

This file is part of OpenCPI <http://www.opencpi.org>

OpenCPI is free software: you can redistribute it and/or modify it under the
terms of the GNU Lesser General Public License as published by the Free Software
Foundation, either version 3 of the License, or (at your option) any later
version.

OpenCPI is distributed in the hope that it will be useful, but WITHOUT ANY
WARRANTY; without even the implied warranty of MERCHANTABILITY or FITNESS FOR A
PARTICULAR PURPOSE. See the GNU Lesser General Public License for more details.

You should have received a copy of the GNU Lesser General Public License along
with this program. If not, see <http://www.gnu.org/licenses/>.
\fi

%----------------------------------------------------------------------------------------
% Update the docTitle and docVersion per document
%----------------------------------------------------------------------------------------
\def\docTitle{OpenCPI\\ Rx App Guide\\ (E310 Supplement)}
\def\docVersion{1.4}
%----------------------------------------------------------------------------------------
\input{../../../snippets/LaTeX_Header.tex}
\date{Version \docVersion} % Force date to be blank and override date with version
\title{\docTitle}
\lhead{Rx App Guide}
%----------------------------------------------------------------------------------------
%\usepackage[T1]{fontenc} % http://tex.stackexchange.com/a/181119
\usepackage{graphicx}
\graphicspath{ {figures/} }
\usepackage{textcomp}

\begin{document}
\maketitle
%\thispagestyle{fancy}
\newpage
	\begin{center}
	\textit{\textbf{Revision History}}
		\begin{table}[H]
		\label{table:revisions} % Add "[H]" to force placement of table
			\begin{tabularx}{\textwidth}{|c|X|l|}
			\hline
			\rowcolor{blue}
			\textbf{Revision} & \textbf{Description of Change} & \textbf{Date} \\
		    \hline
		    v1.1 & Initial Release & 3/2017 \\
		    \hline
		    v1.2 & Updated for OpenCPI Release 1.2 & 8/2017 \\
			\hline
			v1.3 & Updated for OpenCPI Release 1.3 & 1/2018 \\
			\hline
			v1.3.1 & Updated for OpenCPI Release 1.3.1, including FMCOMMS2/3 support & 3/2018 \\
			\hline
			v1.3.1-E3XX & Updated for E310 support & 3/2018 \\
			\hline
			v1.4 & Updated with simplications and references to assets' document & 9/2018 \\
			\hline
			\end{tabularx}
		\end{table}
	\end{center}

\newpage
\tableofcontents
\pagebreak
\vspace{1pc}
\hrule
\def\assetsdoc{\noindent For more information on this application, see \code{ocpi.assets}'s more in-depth version of the \textit{RX\_app} document.}
\section{Document Scope}
This document describes the ANGRYVIPER Receive demo application or ``Rx App". It includes a description of the RX App application and instructions on how to setup the various supported hardware platforms, build and execution of the application.

\section{Description}
\assetsdoc
\section{Building the Application}
\subsection{Common Application Worker Dependencies}
The following application workers, sorted by component library name, must be built prior to building the RX application assembly. See Appendix A for the parameter configurations used in the application, and see the individual component datasheets for more information.\par\bigskip
	\begin{minipage}[t]{.5\textwidth}
	\begin{itemize}
		\item ocpi.core
			\subitem file\_write.rcc
		\item ocpi.assets.util\_comps
			\subitem timestamper.hdl
		\item ocpi.assets.dsp\_comps
			\subitem cic\_dec.hdl
			\subitem complex\_mixer.hdl
			\subitem iq\_imbalance\_fixer.hdl
			\subitem dc\_offset\_filter.hdl
	\end{itemize}
	\end{minipage}

\subsection{Hardware-Specific Worker Dependencies}
The following workers, sorted by component library name, must be built prior to building RX app. See Appendix A for the parameter configurations used in the application, and see the individual component datasheets for more information and build instructions.\par\bigskip
	\begin{minipage}[t]{\textwidth}
	\textbf{E310}
	\begin{itemize}
		\item ocpi.bsp.e310.cards
			\subitem e3xx\_mimo\_xcvr\_filter\_proxy.rcc
			\subitem e3xx\_mimo\_xcvr\_filter.hdl
			\subitem e3xx\_i2c.hdl
			\subitem e3xx\_rx.rcc
			\subitem e3xx\_mimo\_xcvr\_ad5662.hdl
	\end{itemize}
	\end{minipage} \medskip

\newpage

\subsection{HDL Assembly and HDL Container}
\assetsdoc
\subsection{Performance and Resource Utilization}
\begin{scriptsize}
\begin{tabular}{|P{3.1cm}|c|P{1.5cm}|c|c|c|c|}
\hline
\rowcolor{blue}
Hardware Configuration & FPGA                & Registers (typ) & LUTs (typ) & Fmax (typ)   & Memory/Special Functions & Design Suite       \\
\hline\
Matchstiq-Z1               & XC7Z020-1-CLG484    & 9813     & 9918 & 100 MHz & \begin{tabular}{@{}c@{}}DSP48E1s=15 (7\%) \\ RAMB36E1s=21 \\ BUFGs=2 \\ BUFGCTRLs=2\end{tabular} & Vivado 2017.1 \\
\hline\
ML605/FMCOMMS2/3 HPC   & XC6VLX240T-1-FF1156 & 13,972 (4\%) & 18,149 (12\%) & 125 MHz & \begin{tabular}{@{}c@{}}DSP48E1s=15 (1\%) \\ BUFG/BUFGCTRLs=6\end{tabular} & ISE 14.7           \\
\hline
ML605/FMCOMMS2/3 LPC   & XC6VLX240T-1-FF1156 & 13,972 (4\%) & 18,103 (12\%) & 125 MHz & \begin{tabular}{@{}c@{}}DSP48E1s=15 (1\%) \\ BUFG/BUFGCTRLs=6\end{tabular} & ISE 14.7           \\
\hline
ML605/Zipper HPC       & XC6VLX240T-1-FF1156 & 13850     & 18325 & 125 MHz & \begin{tabular}{@{}c@{}}DSP48E1s=15 (1\%) \\ BUFG/BUFGCTRLs=7\end{tabular}                  & ISE 14.7           \\
\hline
ML605/Zipper LPC       & XC6VLX240T-1-FF1156 & 13850     & 18404 & 125 MHz & \begin{tabular}{@{}c@{}}DSP48E1s=15 (1\%) \\ BUFG/BUFGCTRLs=7\end{tabular}                  & ISE 14.7           \\
\hline
Stratix IV/Zipper HSMCA& EP4SGX230K-C2-F40   & 41017     & 27147 & 125 MHz & \begin{tabular}{@{}c@{}}DSP18x18s=26 (2\%) \\ GXB Receiver PCSs=4 \\ GXB Transmitter PCSs=4 \\ GXB Receiver PMAs=4 \\ GXB Transmitter PMAs=4 \\ PLLs=1\end{tabular}           & Quartus 15.1.0 \\
\hline
Stratix IV/Zipper HSMCB& EP4SGX230K-C2-F40   & 41017     & 27147 & 125 MHz & \begin{tabular}{@{}c@{}}DSP18x18s=26 (2\%) \\ GXB Receiver PCSs=4 \\ GXB Transmitter PCSs=4 \\ GXB Receiver PMAs=4 \\ GXB Transmitter PMAs=4 \\ PLLs=1\end{tabular}           & Quartus 15.1.0 \\
\hline
Zedboard/FMCOMMS2/3    & XC7Z020-1-CLG484    & 8,882 (9\%) & 8,865 (17\%) & 100 MHz & \begin{tabular}{@{}c@{}}DSP48E1s=15 (7\%) \\ RAMB36E1s=18 \\ BUFGs=1 \\ BUFGCTRLs=2\end{tabular}& Vivado 2017.1 \\
\hline
Zedboard/Zipper        & XC7Z020-1-CLG484    & 8759     & 8872 & 100 MHz & \begin{tabular}{@{}c@{}}DSP48E1s=15 (7\%) \\ RAMB36E1s=21 \\ BUFGs=2 \\ BUFGCTRLs=2\end{tabular}& Vivado 2017.1 \\
\hline
E310                   & XC7Z020-1-CLG484    & 9560     & 9357 & 100 MHz & \begin{tabular}{@{}c@{}}DSP48E1s=15 (7\%) \\ RAMB36E1s=18 \\ BUFGs=1 \\ BUFGCTRLs=1\end{tabular}& Vivado 2017.1 \\
\hline
\end{tabular}
\end{scriptsize}

\subsection{Executable}
\noindent To build for the E310 (which runs the xilinx13\_4 PetaLinux operating system), run the following command from the \textit{rx\_app} directory:\par\medskip
\texttt{ ocpidev build --rcc-platform xilinx13\_4 }\par\medskip
\assetsdoc
\section{Testing the Application}
\subsection{Sample Test Setup}
\assetsdoc
\subsection{Artifacts}
\assetsdoc \\
\noindent Appendix B includes a list of the artifacts required for each platform and mode.
\subsection{Arguments to executable}
\assetsdoc
\subsection{Library Path Requirements}
\noindent Prior to running the application, the environment variable OCPI\_LIBRARY\_PATH must be configure, such that, all of the Rx application's run-time artifacts can be located. OpenCPI conveniently provides access to a project's run-time artifacts at the top-level of each project in a directory called artifacts. Reference the OpenCPI Application Development Guide for more about OCPI\_LIBRARY\_PATH. \par\medskip

\noindent The following are recommendations for configuring the OCPI\_LIBRARY\_PATH based on the platform, the use of a daughter card and specific slot that card is installed. For all recommendations:
\begin{itemize}
  \item All paths are relative to the applications/rx\_app/ directory.\\
\end{itemize}

\noindent\textbf{Recommended Library Path for E310}\\
Note: assumes running from the e310 project \\
\noindent
\verb|OCPI_LIBRARY_PATH=../../hdl/cards/lib:| \\
\verb|../../ocpi.bsp.e310.dc_offset_iq_imbalance_mixer_cic_dec_timestamper_e3xx_cfg| \\
\verb|_1rx_0tx_mode_2_cmos_cnt_1rx_0tx_thruasm_mode_2_cmos_e3xx.hdl.0.e3xx.gz:| \\
\verb|<path-to-core-project>/exports/artifacts:| \\
\verb|<path-to-assets-project>/exports/artifacts| \\
\par\medskip
\pagebreak
\subsection{Expected results}
\noindent A python script is included with the application for plotting the received data in both the time and frequency domain. Using the test setup shown above and the default settings for the GNUradio block diagram, run the application with the following arguments:\par\medskip

\small

\noindent\textbf{E310}
\scriptsize
\noindent
\begin{verbatim}
# Usage is:
# ./target-xilinx13_4/rx_app rf_tune_freq data_bw rf_bw rf_gain bb_bw bb_gain if_tune_freq runtime enable_timestamps frontend
  ./target-xilinx13_4/rx_app 1000         0.512   -1    24      1     -1      0.256        1       1                 e3xx
\end{verbatim}
\small
\par\medskip
\noindent The output file can then be plotted with the python script with the following syntax and the output can be seen below:\par\medskip
\noindent\texttt{python ./scripts/plotAndFftAndTime.py odata/rx\_app\_raw.out complex 18000 256000 16352}\par
	\begin{figure}[h]
	 	\centering
		\includegraphics[scale=.2]{rx_app_iq_plot}
		\label{fig:rx_app_iq_plot}
	\end{figure}
	\begin{figure}[h]
	 	\centering
		\includegraphics[scale=.2]{rx_app_fft_plot}
		\caption{Output of RX app}
		\label{fig:rx_app_fft_plot}
	\end{figure}
\noindent Alternatively, the shortened file can be plotted which will ignore optionally unwanted startup data:\par\medskip
\noindent\texttt{python ./scripts/plotAndFftAndTime.py odata/rx\_app\_shortened.out complex 18000 256000 16352}\par\medskip
\noindent The default sample rate for the GNUradio block diagram is 512 kS/s. It is recommended that when using RX app with this input signal that a sample rate close to 512 kS/s be used. Higher sample rates are still valid, but may produce plots that look drastically different than those shown here.\par\medskip
\newpage
\noindent Timestamps are embedded, optionally, in the output file, and in addition to plotting, the script parses out and prints the timestamps. An example output gathered using the syntax above:\par\medskip
\scriptsize\noindent\texttt{Timestamp at index: 000000000 :  1.0728292 Seconds: 0x1 Fraction: 0x12a4eec4  \\
Timestamp at index: 000008180 :  1.0887978 Seconds: 0x1 Fraction: 0x16bb73ba ('Delta: 0.0159686', 'Expected:, 0.0159688')\\
Timestamp at index: 000016360 :  1.1047664 Seconds: 0x1 Fraction: 0x1ad1f906 ('Delta: 0.0159686', 'Expected:, 0.0159688')}\par\medskip
\noindent\small A small discrepancy (+/- 10) between Delta and Expected is typical. The difference is an artifact of the resolution of the fractional part of the timestamp applied in the timestamper HDL component. More information can be found in the timestamper component datasheet.\par\medskip
\par\medskip

\subsection{Using a RF Signal Generator}
\noindent As mentioned earlier, an arbitrary RF signal generator can be used with RX app instead of the Ettus N210. Below is an example using a signal generator and a E310.\par\medskip
\noindent In this example, the signal generator is set to 1.001250 GHz with an amplitude of -40 dBm for the E310. The following parameters can be passed to the executable:\par\medskip

\small
\normalsize
\noindent\textbf{E310}\\
\scriptsize
\noindent
\begin{verbatim}
# Usage is:
# ./target-xilinx13_4/rx_app rf_tune_freq data_bw rf_bw rf_gain bb_bw bb_gain if_tune_freq runtime enable_timestamps frontend
  ./target-xilinx13_4/rx_app 1000         2.5     -1    24      2.5   -1      1            1       1                 e3xx
\end{verbatim}
\par\medskip
\small
\noindent Here we plot output data:\par\medskip
\noindent\texttt{python ./scripts/plotAndFftAndTime.py odata/rx\_app\_shortened.out complex 65536 2500000 16352}\par\medskip
        \begin{figure}[H]
                \centering
                \includegraphics[scale=.5]{rx_app_sig_gen_time_domain}
                \label{fig:rx_app_sig_gen_time_domain}
        \end{figure}
        \begin{figure}[H]
                \centering
                \includegraphics[scale=.5]{rx_app_sig_gen_fft}
                \caption{Output of RX app}
                \label{fig:rx_app_sig_gen_fft}
        \end{figure}
\subsection{Known Issues}
\noindent
\begin{itemize}
  \item If the path \path{/var/volatile} does not exist or requires root permission to write to, you will need to modify the ACI and the application XML to use a different directory for writing data. This involves simply finding and replacing \path{/var/volatile} with a different directory in the \path{.cxx} and \path{.xml} files. Failing to make this change when necessary may result in a segmentation fault error at application runtime.
\end{itemize}
% AV-3179
\section{Appendix A: Worker Parameters}
\begin{minipage}[t]{.5\textwidth}
	\textbf{E310}
	\begin{itemize}
		\item cic\_dec.hdl
			\subitem N = 3
			\subitem M = 1
			\subitem R = 8
			\subitem DIN\_WIDTH = 16
			\subitem ACC\_WIDTH = 25
			\subitem DOUT\_WIDTH = 16
		\item complex\_mixer.hdl
			\subitem NCO\_DATA\_WIDTH\_p = 12
			\subitem INPUT\_DATA\_WIDTH\_p = 12
			\subitem CORDIC\_STAGES\_p = 16
			\subitem PEAK\_MONITOR\_p = true
		\item iq\_imbalance\_fixer.hdl
			\subitem DATA\_WIDTH\_p = 16
			\subitem ACC\_PREC\_p = 34
			\subitem PEAK\_MONITOR\_p = true
		\item dc\_offset\_filter.hdl
			\subitem DATA\_WIDTH\_p = 16
			\subitem PEAK\_MONITOR\_p = true
		\item e3xx\_i2c.hdl
			\subitem CP\_CLK\_FREQ\_p = 100e6
			\subitem FMC\_GA1 = 0
			\subitem FMC\_GA0 = 0
		\item ad9361\_spi.hdl
			\subitem CP\_CLK\_FREQ\_HZ\_p = 100e6
		\item ad9361\_data\_sub.hdl
			\subitem LVDS\_p = false
			\subitem HALF\_DUPLEX\_p = false
			\subitem SINGLE\_PORT\_p = true
			\subitem SWAP\_PORTS\_p = false
			\subitem DATA\_CLK\_Delay = 11
			\subitem RX\_Data\_Delay = 0
			\subitem FB\_CLK\_DELAY = 12
			\subitem TX\_DATA\_DELAY = 0
		\item ad9361\_adc\_sub
			\subitem LVDS\_p = false
			\subitem HALF\_DUPLEX\_p = false
			\subitem SINGLE\_PORT\_p = true
		\item ad9361\_dac\_sub
			\subitem LVDS\_p = false
			\subitem HALF\_DUPLEX\_p = false
			\subitem SINGLE\_PORT\_p = true
	\end{itemize}
\end{minipage} \newpage
\section{Appendix B: Artifacts}
\subsection{E310}
	\begin{itemize}
	\item dc\_offset\_iq\_imbalance\_mixer\_cic\_dec\_timestamper\_e3xx\_cfg\_1rx\_0tx\_mode\_2\_cmos\_cnt\_1rx\_0tx\_thruasm\_mode\_2\_cmos\_e3xx.bitz
	\item target-xilinx13\_4/file\_write\_s.so
	\item target-xilinx13\_4/e3xx\_rx\_s.so
	\item target-xilinx13\_4/e3xx\_mimo\_xcvr\_filter\_proxy\_s.so
	\item target-xilinx13\_4/ad9361\_config\_proxy\_s.so
	\end{itemize}
\end{document}
